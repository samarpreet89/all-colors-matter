\documentclass[10pt,twocolumn,letterpaper]{article}

\usepackage{cvpr}
\usepackage{times}
\usepackage{epsfig}
\usepackage{graphicx}
\usepackage{amsmath}
\usepackage{amssymb}
\usepackage{multirow}

% Include other packages here, before hyperref.

% If you comment hyperref and then uncomment it, you should delete
% egpaper.aux before re-running latex.  (Or just hit 'q' on the first latex
% run, let it finish, and you should be clear).
\usepackage[breaklinks=true,bookmarks=false]{hyperref}

\cvprfinalcopy % *** Uncomment this line for the final submission

\def\cvprPaperID{****} % *** Enter the CVPR Paper ID here
\def\httilde{\mbox{\tt\raisebox{-.5ex}{\symbol{126}}}}

% Pages are numbered in submission mode, and unnumbered in camera-ready
\ifcvprfinal\pagestyle{empty}\fi
%\setcounter{page}{4321}
\begin{document}

%%%%%%%%% TITLE
\title{A convolutional classification approach to colorization}

\author{Vincent Billaut\\
Department of Statistics\\
Stanford University\\
{\tt\small vbillaut@stanford.edu}
% For a paper whose authors are all at the same institution,
% omit the following lines up until the closing ``}''.
% Additional authors and addresses can be added with ``\and'',
% just like the second author.
% To save space, use either the email address or home page, not both
\and
Matthieu de Rochemonteix\\
Department of Statistics\\
Stanford University\\
{\tt\small mderoche@stanford.edu}
\and
Marc Thibault\\
ICME\\
Stanford University\\
{\tt\small marcthib@stanford.edu}
}

\maketitle
%\thispagestyle{empty}

%%%%%%%%% ABSTRACT

% TODO for final report

\begin{abstract}
Insert abstract here
\end{abstract}

%%%%%%%%% BODY TEXT
\section*{Introduction}

The problem of colorization is one that comes quickly to mind when thinking about interesting challenges involving pictural data. Namely, the goal is to build a model that takes the greyscale version of an image (or even an actual ``black and white'' picture) and outputs its colorized version, as close to the original as possible (or at least realistic, if the original is not in colors). This problem is complex and interesting for several reasons, as the final output needs to be an image of the same dimension as the input image. We want to train a model that is able to recognize shapes that are typical of a category of items and apply the appropriate colorization. 

One clear upside to this challenge is that any computer vision dataset, and even any image bank really, is a proper dataset for the colorization problem (the image itself is the model's expected output, and its greyscale version is the input to the model).
The input to our algorithm is simply an image in grayscale, and the output is the same image, colorized. A conversion of the images to the YUV format allows an easy formulation of the problem in terms of a reconstitution of the U and V channels. 

We formulate the colorization problem as a classification problem to gain flexibility in the prediction process and output more colorful images. We aim at reproducing state of the art results that give vivid, visually appealing results, with a much smaller network.  

\section{Related Work} \label{relatedwork}

Historically, older approaches of the colorization problem use an idditional input, a \textit{seed scribble} from the user to propagate colors, as does \cite{levin2004colorization}. It may also be seen as a corollary of color style transfer algorithms using a similar image as a "seed" as in \cite{he2017neuralct}. Here, we are interested in fully automatic recolorization

Classical approaches to this task, \eg \cite{cheng2015deep} and \cite{dahl2016tinyclouds}, aim at predicting an image as close as possible to the ground truth, and notably make use of a simple $L_2$ loss, which penalizes predictions that fall overall too far from the ground truth. As a consequence, the models trained following such methods usually tend to be very conservative, and to give desaturated, pale results. Usually, the images need some postprocessing adjustments as in \cite{deshpande2015learning} to have a realistic aspect.

On the contrary, authors of \cite{zhang2016colorful} take another angle and set their objective to be ``\textit{plausible} colorization'' (and not necessarily \textit{accurate}), which they validate with a large-scale human trial. To achieve such results, they formulate the colorization task as a classification task using color bins, as suggested already in \cite{charpiat2008automatic}. 

Their approach is the most appealing as they have colorful results and we found the classification formulation to be interesting. However, the network architecture they use is very heavy (taking more than 20 GB of memory), and the scale of their training set (several millions of images). The reason behind this is that to encode meaningful fetures that help to colorize the image, one needs to have a large spatial receptive field. THe approach of the article is to downsample the image a lot in the layers in the middle and then upsample using Convolutional Transpose layers. To keep a lot of information in the intermediate layers, the number of filters in the model of \cite{zhang2016colorful} is very large, resulting in a model that is large and expensive to train.

Authors of \cite{ronneberger2015unet} have shown that connections between hidden layers of a bottleneck neural network could enhance the performance greatly, by helping the upsampling process and improving the gradient flow. 

We hope that applying this method will allow us to train a colorizing model more quickly and more efficently, with less parameters. 

Part of the challenges that are interesting but not yet tackled in the literature involve videos. General information propagation frameworks in a video involving bilateral netwoks as discussed in \cite{jampani2017video} could be seen as a a potential way to implement cosistent colorization of video sequences. The work realized in \cite{zhu2017video} is also interesting since it tackles video stylization by grouping the frames, choose a representative frame for each group and use the output of the network on this frame as a guideline, which enhances temporal consistency. However, the adaptation of such an algorithm to the much more complex task of image colorization is far beyond the scope of this project. 

Actually, one promising way to perform image colorization is to be able to learn meaningful color-related representations for the images (which often involves using very deep and heavy or pretrained architecture as in \cite{larsson2016repres}) and then ensure the temporal consistency of them. 

Given our commitment to developing a lightweight model, we prefered focusing on an efficient colorization for images and then add alayer of temporal convolution for videos. 


\section{Methods}
\section{Dataset and Features}
\section{Results and discussion}
\section{Conclusion and perspectives}
\section*{Contributions and acknowledgements}


{\small
\bibliographystyle{ieee}
\bibliography{references}
}

\end{document}
